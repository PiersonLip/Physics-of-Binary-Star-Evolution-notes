\documentclass{article}
\usepackage{darkmode}
\usepackage{amsmath,amssymb}
\usepackage[colorlinks]{hyperref}
\usepackage[automake]{glossaries-extra} % <-- builds glossary automatically with latexmk/Overleaf

\title{Default Title}
\author{Pierson Lipschultz}

\enabledarkmode

% GOOD: label has no spaces; display text goes in name=
\newglossaryentry{algol-type}{
  name={Algol Type},
  description={Type of eclipsing binary with properties similar to the Algol system. It appears paradoxical because the more evolved star has a smaller mass, explained by mass transfer.}
}


\makeglossaries % harmless with glossaries-extra; required if you switch to glossaries

\begin{document}
\tableofcontents

\section{Intro}
\subsection{Importance of binaries}
\begin{list}{-}{}
\item Plays a key role in the evo of massive stars
\item first presumed to exist because of \gls{algol-type} stars. These stars were explained with mass transfer
\item source of strongest GW rad, GrBs, and have r-process elements.
\item Likely cause of SNe Ia, Ib, Ic (SN with a lack of hydrogen in their spectra) 
\item cause of low-intermediate mass stars with odd chemical compositions, ex. barium stars
\end{list}

\subsection{Kilonova}
\begin{list}{-}{}
\item allows exact placing of a merger in the sky
\end{list}

\subsection{XrBs}
\begin{list}{-}{}
\item Without mass transfer, most systems would not survive the SN of the primary, instead becoming unbound
\end{list}

\clearpage

\printglossaries
\end{document}
