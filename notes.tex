\documentclass{article}
\usepackage{darkmode}
\usepackage{amsmath,amssymb}
\usepackage{gensymb}
\usepackage[colorlinks]{hyperref}
\usepackage[automake]{glossaries-extra} % <-- builds glossary automatically with latexmk/Overleaf

\title{Default Title}
\author{Pierson Lipschultz}

\enabledarkmode

% GOOD: label has no spaces; display text goes in name=
\newglossaryentry{algol-type}{
  name={Algol Type},
  description={Type of eclipsing binary with properties similar to the Algol system. It appears paradoxical because the more evolved star has a smaller mass, explained by mass transfer.}
}

\newglossaryentry{kilonova}{
name={Kilonova},
description={A merger of either a NS+NS (DNS) binary or a NS+BH binary. Results in a bright signal resulting from the rapid decay the the NS material. Results in a peak brightness around 1000x that of a standard nova, hence the name. Likely a standard candle.}
}

\newglossaryentry{proper-motion}{
name = {Proper Motion},
description = {The motion of a star in the sky relative to more distant stars. Allows us understand the velocity of a star relative to the earth}
}

\makeglossaries % harmless with glossaries-extra; required if you switch to glossaries

\begin{document}
\tableofcontents

\section{Intro}
\noindent \textbf{\large Importance of binaries}
\begin{list}{-}{}
\item Plays a key role in the evolution of massive stars
\item first presumed to exist because of \gls{algol-type} stars. These stars were explained with mass transfer
\item Mergers are a key source of strongest GW rad, GrBs, and have r-process elements. (i.e. in \gls{kilonova}) 
\item Likely cause of SNe Ia, Ib, Ic (SN with a lack of hydrogen in their spectra) 
\item cause of low-intermediate mass stars with odd chemical compositions, ex. barium stars. 
\end{list}

\section{Visual binaries and the universal validity of the laws of physics}

\subsection{Visual binaries}
\begin{list}{-}{}

\item Visual binary \(\rightarrow\) two in proximity through a telescope

\item Physical \(\rightarrow\) gravitonalty bound orbits

\item Optical \(\rightarrow\) happen to be close together in the sky

\end{list}

\begin{list}{-}{}
\item Visual binaries have longer orbits
\item the proof of physical binaries (and not just coincidence) was proved in 1767 by John Michell
\item Shows that Newton's law of gravity applied outside just our solar system
\end{list}
\subsection{Astrometric Binaries}
\noindent \textbf{\large Wide binaries} 
\begin{list}{-}{}
\item Binaries where one of the stars is too faint to directly observe
\item Can still be detected by the following \gls{proper-motion} of the visible component
\item Center of mass moves along, star orbits this center, showing a periodic wiggle in its apparent motion (I wonder how this can be muddled with and/or separated from parallax. I'm guessing they are in different planes?? (next graphic helped(it wiggles slightly due to yearly parallax, however, the proper motion is much greater and over a much longer duration, so it sort of just becomes ``noise'' relative)))
\end{list}
\subsection{Spectroscopic Binaries}
\noindent \textbf{\large Spectroscopic Binaries}
\begin{list}{-}{}
\item A binary which is determined to be a binary due to its spectra
\item double-lined binary (SB2) \(\rightarrow\) when both sets of spectral lines can be observed redshifting and blueshifting
\item single-lined (SB1) \(\rightarrow\) when only set of spectral lines can be observed redshifting/blueshifting
\item this is due to Doppler shifting when one of the binaries is relativistically coming towards us and the other away
\item Doppler shift (\(\Delta\lambda\)) can be related to the radial velocity (the component of the velocity on our line of sight)
\item \[\frac{\Delta\lambda}{\lambda} = \frac{v_{rad}}{c}\]
\end{list}
\subsection{Eclipsing Binaries}
Binaries where one star passes in front of the other at some point during its period
\begin{list}{-}{}
\item A partial eclipse will happen if  
\item \[\sin(\phi) < \sin(\phi_0) = \frac{R_1+R_2}{a}\]
\item where inclination \(i\) is defined as the angle between the orbital plane and the plane of the sky \(\rightarrow\) \(\phi = 90 \degree - i\) and \(a\) is the orbital radius  
\item and a total eclipse will occur if 
\item \[\sin(\phi) < \sin(\phi_1) = \frac{R_1-R_2}{a}\]
\item We assume that the angles between line of sight and orbital planes are random, the probability of having an angle between 0 and \(\phi_0\) is the fraction of the surface area of the half sphere between the pole and the circle at an angle \(\phi_0\) from the pole, which is equal to \(2\pi(1-\cos(\phi_0))\). While the area of the half sphere is \(2\pi\)\footnote{Try this myself and verify that this makes sense}. Thus, the probability of having an eclipse is 
\item \[\mathcal{F}_{eclipse} = (1-\cos\phi_0)\]
\item Assuming \(\phi_0\) small
\item \[\sin\phi \simeq \phi_0 \simeq \frac{(R_1 +R)_2}{a}\]
\item due to series expansion\footnote{for sure do this at some point}
\item \[\cos\phi_0 \simeq 1 - \frac{(R_1 + R_2)^2}{2a^2}\]
\item Thus
\item \[\mathcal{F}_{eclipse} =\frac{(R_1 + R_2)^2}{2a^2} \]
\item Because \(R_1 +R_2 < a\), eclipses are most likely for either dwarf stars with a  very close orbital radius, or wide binaries where at least one star is a giant 
\end{list}

\subsection{}
\begin{list}{-}{}
\item It is possible for a spinning WDs to show variation (but not pulses) in star brightness
\item This is because of the magnetic fields on WDs causing bright spots
\item Pulsation can be determined by the density 
\end{list}
\noindent \textbf{\large Novae/CVs}
\begin{list}{-}{}
\item Caused my mass transfer onto a WD
\item WD shell goes SN, causing Novae
\item Standard candle
\item Has a variation called dwarf Novae 
\item caused by instability in the accretion disk
\item much weaker
\end{list}

\subsection{}
\begin{list}{-}{}
\item Brightest sources of x-rays were found to be CV accreting systems
\end{list}

\subsection{}
\begin{list}{-}{}
\item The first NS star XrB
\item Regular periodicity of with increase of x-ray emission
\item Caused by the NS being obscured by the larger star
\end{list}

\subsection{First BH XrB}
\begin{list}{-}{}
\item Discovered x-ray source with nearby supergiant
\item Falls into two classes
\item HMXBs and LMXBs
\item determined by the mass of the acceptor with respect to the donor
\item x-ray emission called by infilling matter
\item \(\frac{GMm}{R} = .01mc^2\)\footnote{Shockingly simple eq}
\item much much much more efficient than any fusion reactor on earth
\item values as high as \(.42mc^2\)
\item Common in Globular clusters
\end{list}

\subsubsection{Double NSs \& BHs}
\begin{list}{-}{}
\item Proved by GW detection
\item 
\end{list}

\clearpage

\printglossaries
\end{document}
