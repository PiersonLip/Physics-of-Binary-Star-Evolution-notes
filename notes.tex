\documentclass{article}
\usepackage{darkmode}
\usepackage{amsmath,amssymb}
\usepackage{gensymb}
\usepackage[colorlinks]{hyperref}
\usepackage[automake]{glossaries-extra}

\title{Default Title}
\author{Pierson Lipschultz}

\enabledarkmode


\newglossaryentry{algol-type}{
  name={Algol Type},
  description={Type of eclipsing binary with properties similar to the Algol system. It appears paradoxical because the more evolved star has a smaller mass, explained by mass transfer.}
}

\newglossaryentry{kilonova}{
name={Kilonova},
description={A merger of either a NS+NS (DNS) binary or a NS+BH binary. Results in a bright signal resulting from the rapid decay the the NS material. Results in a peak brightness around 1000x that of a standard nova, hence the name. Likely a standard candle.}
}

\newglossaryentry{proper-motion}{
name = {Proper Motion},
description = {The motion of a star in the sky relative to more distant stars. Allows us understand the velocity of a star relative to the earth}
}

\newglossaryentry{MSP}{
name = {MSP},
description = {Milisecond Pulsar. \\ A pulsar with a spin period \(\sim 30\)ms}
}

\newglossaryentry{GWR}{
name = {GWR},
description = {gravitnal wave radiation.\\ Process of stars losing angular momentum through the radiation of gravitnal waves}
}

\newglossaryentry{MB}{
name = {MB},
description = {Magnetic Breaking. \\ Process where two stars ``couple'', causing a loss of angular momentum}
}

\newglossaryentry{Recylced-Pulsar}{
name = {Recylced Pulsar},
description = {A pulsar which obtained its rapid spin through accretion. Has weaker magnetic fields then newly formed NSs }
}

\newglossaryentry{TI-SNe}{
name = {Type I SNe},
description = {A SNe \textbf{without} hydrogen in its spectra\\ Generally standard candles. They are thermonuclear explosions of carbon-oxygen WDs. See section(\ref{WD-SNe})}
}


\newglossaryentry{TIb-SNe}{
name = {Type I SNe},
description = {A SNe \textbf{without} hydrogen and with helium in its spectra. \\ Caused by collaspe of the naked helium core star.}
}


\newglossaryentry{TIc-SNe}{
name = {Type I SNe},
description = {A SNe \textbf{without} hydrogen or helium in its spectra. \\ A star where both the hydrogen and helium envelopes have been fully stripped.}
}


\newglossaryentry{TII-SNe}{
name = {Type II SNe},
description = {A SNe \textbf{with} hydrogen in its spectra \\ generally found in galaxies with high star formation rate. Likely caused by core collaspe}
}

\newglossaryentry{WR}{
name = {WR},
description = {Wolf-Rayet star. \\A star so massive (generally more than \(\sim25M_\odot\)) that it sheds its hydrogen layer due to stellar wind. It may also shed its helium layer as well.}
}

\newglossaryentry{Chandrasekhar-limit}{
name = {Chandrasekhar limit},
description = {Maxium mass of a WD, \(\sim1.4M_\odot\)}
}

\newglossaryentry{edge-limit-detonation}{
name = {edge limit detonation},
description = {A WD SNe trigged by accreted H/He on the suface which sends a shockwave into the star, trigging runaway fusion.}
}

\newglossaryentry{SD}{
name = {SD},
description = {\textit{Single Degenerate}, donor is a normal star, accretor is WD}
}

\newglossaryentry{DD}{
name = {DD},
description = {\textit{Double Degenerate}, donor as well as accretor are WDs}
}

\newglossaryentry{Blue-stragglers}{
name = {Blue stragglers},
description = {}
}

\newglossaryentry{Barium-stars}{
name = {Barium stars},
description = {G and K type gaints with a `overabundance' of barium and other s-process elements \\Wide binaries with longer periods}
}

\newglossaryentry{WNh}{
name = {WNh},
description = {Very massive stars with strong stellar winds, leading to spectra very similar to that of WRs, but with with hydrogen emission}
}

\newglossaryentry{eddington-limit}{
name = {Eddington Limit},
description = {\(\dot{M}_{Edd} \simeq 1.8 \times 10^{-8} M_\odot \mathrm{yr}^{-1}\) For NS \\ Maxium rate of mass transfer onto a NS or BH }
}

\newglossaryentry{thermal-timescale}{
name = {thermal timescale},
description = {}
}

\newglossaryentry{consv-RLO}{
name = {Conservative Roche lobe overflow},
description = {RLO where the total mass and angular momentum remain constant. Is an example of Huangs \textit{slow mode}}
}

\newglossaryentry{ML-CircumDisk}{
name = {Mass loss from a circumbinary disk}, %circumbinary disk?? i hardly know her!
description = {Ejected mass from mass transfer/winds a ring orbiting the system (a circumbinary disk).\\ Refered to by Huang as an \textit{intermediate mode}. Due to tidal interactions these rings ``extract'' angular momentum from the system }
}

\newglossaryentry{Case-A-RLO}{
name = {Case A RLO},
description = {RLO during donor hydrogen core burning}
}

\newglossaryentry{Case-B-RLO}{
name = {Case B RLO},
description = {}
}

\makeglossaries 

\begin{document}
\tableofcontents

\section{Intro}
\noindent \textbf{\large Importance of binaries}
\begin{list}{-}{}
\item Plays a key role in the evolution of massive stars
\item first presumed to exist because of \gls{algol-type} stars. These stars were explained with mass transfer
\item Mergers are a key source of strongest GW rad, GrBs, and have r-process elements. (i.e. in \gls{kilonova}) 
\item Likely cause of SNe Ia, Ib, Ic (SN with a lack of hydrogen in their spectra) 
\item cause of low-intermediate mass stars with odd chemical compositions, ex. barium stars. 
\end{list}

\section{Visual binaries and the universal validity of the laws of physics}

\subsection{Visual binaries}
\begin{list}{-}{}

\item Visual binary \(\rightarrow\) two in proximity through a telescope

\item Physical \(\rightarrow\) gravitonalty bound orbits

\item Optical \(\rightarrow\) happen to be close together in the sky

\end{list}

\begin{list}{-}{}
\item Visual binaries have longer orbits
\item the proof of physical binaries (and not just coincidence) was proved in 1767 by John Michell
\item Shows that Newton's law of gravity applied outside just our solar system
\end{list}
\subsection{Astrometric Binaries}
\noindent \textbf{\large Wide binaries} 
\begin{list}{-}{}
\item Binaries where one of the stars is too faint to directly observe
\item Can still be detected by the following \gls{proper-motion} of the visible component
\item Center of mass moves along, star orbits this center, showing a periodic wiggle in its apparent motion (I wonder how this can be muddled with and/or separated from parallax. I'm guessing they are in different planes?? (next graphic helped(it wiggles slightly due to yearly parallax, however, the proper motion is much greater and over a much longer duration, so it sort of just becomes ``noise'' relative)))
\end{list}
\subsection{Spectroscopic Binaries}
\noindent \textbf{\large Spectroscopic Binaries}
\begin{list}{-}{}
\item A binary which is determined to be a binary due to its spectra
\item double-lined binary (SB2) \(\rightarrow\) when both sets of spectral lines can be observed redshifting and blueshifting
\item single-lined (SB1) \(\rightarrow\) when only set of spectral lines can be observed redshifting/blueshifting
\item this is due to Doppler shifting when one of the binaries is relativistically coming towards us and the other away
\item Doppler shift (\(\Delta\lambda\)) can be related to the radial velocity (the component of the velocity on our line of sight)
\item \[\frac{\Delta\lambda}{\lambda} = \frac{v_{rad}}{c}\]
\end{list}
\subsection{Eclipsing Binaries}
Binaries where one star passes in front of the other at some point during its period
\begin{list}{-}{}
\item A partial eclipse will happen if  
\item \[\sin(\phi) < \sin(\phi_0) = \frac{R_1+R_2}{a}\]
\item where inclination \(i\) is defined as the angle between the orbital plane and the plane of the sky \(\rightarrow\) \(\phi = 90 \degree - i\) and \(a\) is the orbital radius  
\item and a total eclipse will occur if 
\item \[\sin(\phi) < \sin(\phi_1) = \frac{R_1-R_2}{a}\]
\item We assume that the angles between line of sight and orbital planes are random, the probability of having an angle between 0 and \(\phi_0\) is the fraction of the surface area of the half sphere between the pole and the circle at an angle \(\phi_0\) from the pole, which is equal to \(2\pi(1-\cos(\phi_0))\). While the area of the half sphere is \(2\pi\)\footnote{Try this myself and verify that this makes sense}. Thus, the probability of having an eclipse is 
\item \[\mathcal{F}_{eclipse} = (1-\cos\phi_0)\]
\item Assuming \(\phi_0\) small
\item \[\sin\phi \simeq \phi_0 \simeq \frac{(R_1 +R)_2}{a}\]
\item due to series expansion\footnote{for sure do this at some point}
\item \[\cos\phi_0 \simeq 1 - \frac{(R_1 + R_2)^2}{2a^2}\]
\item Thus
\item \[\mathcal{F}_{eclipse} =\frac{(R_1 + R_2)^2}{2a^2} \]
\item Because \(R_1 +R_2 < a\), eclipses are most likely for either dwarf stars with a  very close orbital radius, or wide binaries where at least one star is a giant 
\end{list}

\subsection{}
\begin{list}{-}{}
\item It is possible for a spinning WDs to show variation (but not pulses) in star brightness
\item This is because of the magnetic fields on WDs causing bright spots
\item Pulsation can be determined by the density 
\end{list}
\noindent \textbf{\large Novae/CVs}
\begin{list}{-}{}
\item Caused my mass transfer onto a WD
\item WD shell goes SN, causing Novae
\item Standard candle
\item Has a variation called dwarf Novae 
\item caused by instability in the accretion disk
\item much weaker
\end{list}

\subsection{}
\begin{list}{-}{}
\item Brightest sources of x-rays were found to be CV accreting systems
\end{list}

\subsection{}
\begin{list}{-}{}
\item The first NS star XrB
\item Regular periodicity of with increase of x-ray emission
\item Caused by the NS being obscured by the larger star
\end{list}

\subsection{First BH XrB}
\begin{list}{-}{}
\item Discovered x-ray source with nearby supergiant
\item Falls into two classes
\item HMXBs and LMXBs
\item determined by the mass of the acceptor with respect to the donor
\item x-ray emission called by infilling matter
\item \(\frac{GMm}{R} = .01mc^2\)\footnote{Shockingly simple eq}
\item much much much more efficient than any fusion reactor on earth
\item values as high as \(.42mc^2\)
\item Common in Globular clusters
\end{list}

\subsection{Double NSs \& BHs}
\begin{list}{-}{}
\item Proved by GW detection
\item Detections are dominated by DBHs 
\item DNSs formed through lower mass binaries
\end{list}

\subsection{MSPs}
\begin{list}{-}{}
\item Generally remnants of LMXBs
\item Systems with \gls{MSP}s generally have WD partners
\item Mass-transfer through evolution, or through orbital momentum loss through or GWR
\item The NS is greatly accelerated through accretion
\item Old NSs which evolved through this process are called \textit{\gls{Recylced-Pulsar}s}
\end{list}

\subsection{Results of evolution in binaries}
\begin{list}{-}{}
\item SNe caused by the death of stars more massive than \(\sim8M_\odot\)
\item SNe types
\begin{list}{-}{}
\item \gls{TI-SNe}
\item \gls{TIb-SNe}
\item \gls{TIc-SNe}
\item \gls{TII-SNe} 
\end{list}
\item due to the fact that \gls{TIb-SNe} and \gls{TIc-SNe} are stripped cores, it is incredibly likely that they result from binary systems
\item We may also observe both \gls{TIb-SNe} and \gls{TIc-SNe} in \gls{WR}, however, we have \textbf{not} yet
\item WD SNe (section \ref{WD-SNe}) \textit{require} some sort of mass transfer process
\item The companion donor can be either a standard star (\gls{SD}), or another WD (\gls{DD})
\end{list}

\subsection{Weird stars}
\begin{list}{-}{}
\item Blue stragglers
\item Barium 
\end{list}


\section{Orbits and masses of spectroscopic binaries}
\setcounter{subsection}{3}
\subsection{}
This is generally all math that I either know from astro classes, or is so specific id need to revisit it to use anyway\\

\[\frac{L}{L_\odot} = \frac{M}{M_\odot}^{3.5}, M > 1.3M_\odot\]

\subsection{Very massive stars}
\begin{list}{-}{}
\item Most massive has \(L \sim 10^ L_\odot\)
\item Because of the high (\(\sim50000k\)) temps, they have strong stellar wind, and thus very similar spectra to \gls{WR}, which they are distinguished from by the presence of hydrogen 
\item Denoted as \gls{WNh}
\end{list}

\subsection{Stuff that screws up rad vel curves}
\begin{list}{-}{}
\item Rad Vel curve must be off due to a difference in eccentricity values derived from the rad curve and the light curve
\item Rotation effect
\begin{list}{-}{}
\item when the stars are eclipsing the rotational Doppler shift can be added (if the half spinning away from us is obscured) or subtracted to the shift, leading to different rad velocity measurements. This is called the Rossiter-McLaughlin effect. \footnote{This is actually a super neat and intuitive process which i feel like the book just sorta skips and doesnt explain, maybe wanna write out/make some graphics bout this sometime}
\end{list}
\item presence of gas streams
\begin{list}{-}{}
\item wow they could have said literally anything about what a gas stream is in this case
\item im guessing it means some sort of flow of mass coming off of the binaries, maybe in accretion, maybe in jets???
\end{list}
\item reflection effect
\item heating effect
\begin{list}{-}{}
\item Both the reflection and heating effect are caused by that the stars will cast light on each other, both heating and causing reflection
\end{list}
\item deformation effect
\begin{list}{-}{}
\item the above effects cause the curve to be shifted from the c.m., leading to wonky radial vel curves
\end{list}
\item  this is actually pretty important when it comes to HMXBs and properly measuring NS and BH masses
\end{list}


\subsection{Interacting binaries}
\begin{list}{-}{}
\item \(> 70\%\) of massive o-type stars are members of binaries so close that they'll exchange mass or merge before either star explode as a SN

\end{list}
the rest of this section follows the ``way too niche of math to write down'' thing

\section{Roche equipotential}\footnote{I have a ridiculous amount of notes and a whole essay written about this, so notes here Wil be pretty light}

\begin{list}{-}{}
\item This is some black magic mathmathmatic reasoning 
\item Co-rotating binaries with have a tidal bulge that can be `easily' calculated
\item also love the seemingly kinda personal rant about how the RL diagram is actually wrong, then provides a diagram which is way less intuitive  
\end{list}

\subsection{Mass transfer Galore}
\begin{list}{-}{}
\item Mass transfer through \(L_1\) does not majority effect angular momentum, however, transfer through \(L_2\) does, causing a shrinkage in separation, and typically mergers
\item Different ways of modeling the transfer, depending on how you treat angular momentum with respect to the transfer
\item Orbit widens if the donor is less massive then accreter, and shrinks if the donor is more
\end{list}

\subsection{Types of RLO}
\begin{list}{-}{}
\item \gls{consv-RLO}
\item \gls{ML-CircumDisk}
\end{list}

\subsection{General Notes}
\begin{list}{-}{}
\item Mass transfer onto NS and BH are limited by the \gls{eddington-limit}
\item However, mass transferred off of the donor is at the rate of the evolution of its envelope. (\gls{thermal-timescale})
\item mass transfer off of the donor is much greater (2 to 4 orders of magnitude) then the maximum accretion rate thus, the bulk of the mass will be ``blown away''
\item if more than half of the total mass of the binary system is lost in a SN, the system will become unbounded 
\item What on earth is a applegate mechanism
\item and what is gravitational quadrupole coupling
\item ^^ im pretty sure its just when the oblateness of the star various from some convection stuff, which causes the period to variation causing period variations 
\end{list}

\subsection{Stability Criteria}
\begin{list}{-}{}
\item Depends on the response of the star losing mass as well as the RL
\item more likely to detect it if the mass transfer is stable 
\item orbit always expands for \(q < 1 \) and always shrinks when \(q = 1.28\)
\item Depends on whether the donor contracts or swells upon starting to transfer mass
\item Donors with radiative envelopes (\gls{Case-A-RLO}) or slightly convective envelopes (early \gls{Case-B-RLO}) will typically either shrink or stay the same radius
\item 
\end{list}




\pagebreak
\setcounter{section}{-1}
\section{Misc}
\subsection{WD SNe}\label{WD-SNe}
\begin{list}{-}{}
\item This, simply-ish put, is caused by because a WD cannot reach hydrostatic equilibrium. A pressure increase does \textbf{not} lead to increase and radius, and thus cooling, instead, it just increases heat, and thus fusion, quickly getting out of hand. Can b trigged if the \gls{Chandrasekhar-limit} is crossed, as well as if a layer of H/He on top entities, called a \gls{edge-limit-detonation} or ``sub-Chandrasekhar'' explosions.
\item In \gls{DD} systems the WDs must have a small separation, causing \gls{GWR} to shrink the separation until merger \(\rightarrow\) SNe
\end{list}
\clearpage

\printglossaries
\end{document}
