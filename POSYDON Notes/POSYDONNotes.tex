\documentclass{article}
\usepackage{darkmode}
\usepackage{amsmath,amssymb}
\usepackage{gensymb}
\usepackage[colorlinks]{hyperref}
\usepackage[automake]{glossaries-extra} 


\author{Pierson Lipschultz}

\enabledarkmode
\makeglossaries 

\begin{document}

\section{Evolution}

\begin{list}{-}{}
\item Consists of a large number and variety of ``steps'' or solvers
\item Each time a system evolves through one of steps, or meets some other criteria, it is then ``handed off'' to one of these solvers, which then calculates how it evolves 
\item These steps are largely from MESA (?)
\item 
\end{list}


\noindent \textbf{\large Use}
\begin{list}{-}{}
\item You create a class, like ``pop''
\item onto this class you pass all of your evolution augments, properties, configs, etc
\item this largely boils down to \textit{how} to evolve and \textit{what} to evolve
\item then you ``evolve'' this class 
\item that allows you to use a ``manager'' to view the simulated data. This manager is a subclass called by ``pop''.manager
\item the simulated binaries are stored as objects, where each object is the binary, \textit{which itself} also holds objects for each of the stars 
\end{list}

\noindent \textbf{\large Custom Single Binary Evolution}
\begin{list}{-}{}
\item you can once more create a class with the properties of your system. This includes initial time, masses, states, etc
\item then you pass .evolve to it again
\item then you can repeat above
\end{list}
\printglossaries
\end{document}