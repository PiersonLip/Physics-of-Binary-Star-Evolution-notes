\newglossaryentry{BH_LS}{
name={BH-LS},
description={Black Hole w/Luminous companion}
}

\newglossaryentry{RelativisticBeaming}{
name={Relativistic Beaming},
description={The process of an objects brightness being affected by its relative motion to earth, see\ref{sec:RelBeaming}. Note that this process is in conjuction \textit{with} dopller shift} 
}

\newglossaryentry{BEER}{
name={BEER},
description={BEaming, Ellipsoidal, and Reflection/emission modulation \\ Method for detecing exoplants or blackholes based on variations in an observed star. See \gls{RelativisticBeaming}}  
}

\newglossaryentry{TESS}{
name={TESS},
description={Transiting Expoplanet Survey Satelitte \\ satelitte dedicated to searching for exoplants via transit method, similar to kepler, albiet in a \textit{much} larger region.}
}

\newglossaryentry{algol-type}{
  name={Algol Type},
  description={Type of eclipsing binary with properties similar to the Algol system. It appears paradoxical because the more evolved star has a smaller mass, explained by mass transfer.}
}

\newglossaryentry{kilonova}{
name={Kilonova},
description={A merger of either a NS+NS (DNS) binary or a NS+BH binary. Results in a bright signal resulting from the rapid decay the the NS material. Results in a peak brightness around 1000x that of a standard nova, hence the name. Likely a standard candle.}
}

\newglossaryentry{proper-motion}{
name={Proper Motion},
description={The motion of a star in the sky relative to more distant stars. Allows us understand the velocity of a star relative to the earth}
}
 
\newglossaryentry{MSP}{
name={MSP},
description={Milisecond Pulsar. \\ A pulsar with a spin period \(\sim 30\)ms}
}

\newglossaryentry{GWR}{
name={GWR},
description={gravitnal wave radiation.\\ Process of stars losing angular momentum through the radiation of gravitnal waves}
}

\newglossaryentry{MB}{
name={MB},
description={Magnetic Breaking. \\ Process whbere two stars ``couple'', causing a loss of angular momentum}
}

\newglossaryentry{Recylced-Pulsar}{
name={Recylced Pulsar},
description={A pulsar which obtained its rapid spin through accretion. Has weaker magnetic fields then newly formed NSs }
}

\newglossaryentry{TI-SNe}{
name={Type I SNe},
description={A SNe \textbf{without} hydrogen in its spectra\\ Generally standard candles. They are thermonuclear explosions of carbon-oxygen WDs. See section(\ref{WD-SNe})}
}


\newglossaryentry{TIb-SNe}{
name={Type Ib SNe},
description={A SNe \textbf{without} hydrogen and with helium in its spectra. \\ Caused by collaspe of the naked helium core star.}
}


\newglossaryentry{TIc-SNe}{
name={Type Ic SNe},
description={A SNe \textbf{without} hydrogen or helium in its spectra. \\ A star where both the hydrogen and helium envelopes have been fully stripped.}
}


\newglossaryentry{TII-SNe}{
name={Type II SNe},
description={A SNe \textbf{with} hydrogen in its spectra \\ generally found in galaxies with high star formation rate. Likely caused by core collaspe}
}

\newglossaryentry{WR-star}{
name={WR star},
description={Wolf-Rayet star. \\A helium star of mass They are genally so massive (typically more than \(\sim25M_\odot\)) that it sheds its hydrogen layer due to stellar wind. It may also shed its helium layer as well.}
}

\newglossaryentry{WNh}{
name={WNh},
description={Very massive stars with strong stellar winds, leading to spectra very similar to that of WRs, but also with hydrogen emission. These have ejected their outer shells of hydrogen, and show distinct He and/or C emission, buta also is still doing H fusion in the core}
}



\newglossaryentry{edge-limit-detonation}{
name={edge limit detonation},
description={A WD SNe trigged by accreted H/He on the suface which sends a shockwave into the star, trigging runaway fusion.}
}

\newglossaryentry{SD}{
name={SD},
description={\textit{Single Degenerate}, donor is a normal star, accretor is WD}
}

\newglossaryentry{DD}{
name={DD},
description={\textit{Double Degenerate}, donor as well as accretor are WDs}
}

\newglossaryentry{Blue-stragglers}{
name={Blue stragglers},
description={}
}

\newglossaryentry{Barium-stars}{
name={Barium stars},
description={G and K type gaints with a `overabundance' of barium and other s-process elements \\Wide binaries with longer periods}
}


\newglossaryentry{eddington-limit}{
name={Eddington Limit},
description={\(\dot{M}_{Edd} \simeq 1.8 \times 10^{-8} M_\odot \mathrm{yr}^{-1}\) For NS \\ Maxium rate of mass transfer onto a NS or BH }
}

\newglossaryentry{thermal-timescale}{
name={Thermal Timescale},
description={}
}

\newglossaryentry{consv-RLO}{
name={Conservative Roche lobe overflow},
description={RLO where the total mass and angular momentum remain constant. Is an example of Huangs \textit{slow mode}}
}

\newglossaryentry{ML-CircumDisk}{
name={Mass loss from a circumbinary disk}, %circumbinary disk?? i hardly know her!
description={Ejected mass from mass transfer/winds a ring orbiting the system (a circumbinary disk).\\ Refered to by Huang as an \textit{intermediate mode}. Due to tidal interactions these rings ``extract'' angular momentum from the system }
}

\newglossaryentry{Case-A-RLO}{
name={Case A RLO},
description={RLO during donor hydrogen core burning (Meaning mainsequence) 
\cite{yungelson_elusive_2024}}
}

\newglossaryentry{Case-B-RLO}{
name={Case B RLO},
description={This is where the donor star is \textit{post} hydrogen core burning. Ex. hydrogen \textit{shell} burning 
\cite{yungelson_elusive_2024} \\
See \ref{Yungelson-case-b-rlo} for more info} 
}

\newglossaryentry{Degenerate-matter}{
name={degenerate},
description={Matter which is under the effects of degeneracy from the \gls{pauli-exclusion-principle}. This is very commonly seen in neutron stars, as degeneracy pressure is what keeps them from collasping. \\ 
}
}

\newglossaryentry{pauli-exclusion-principle}{
name={Pauli Exclusion Principle},
description={\fbox{See \ref{pauli-exclusion-principle} }}
}


\newglossaryentry{ZAMS}{
name={ZAMS},
description={Zero age main sequence
This is a star which is has literally just started on the main seqeunce. Hence `zero-age'}
}

\newglossaryentry{antibiotic-index}{
name={antibiotic index},
description={Also called a heart capacity ratio. A very very very simplified and abstract defenition is how much a gas will expand when heated. \url{https://en.wikipedia.org/wiki/Heat_capacity_ratio}}
}

\newglossaryentry{photon-gas}{
name={photon gas},
description={A `gas-like' collection of photons. This means that they have a tempature, pressure, and entropy. \\ They are most common in black-body equilibrums. \url{https://en.wikipedia.org/wiki/Photon_gas}}
}

\newglossaryentry{eddington-lum-limit}{
name={Eddington (Luminosity) Limit},
description={The uper limit of the mass of a star, defined as \(3.2 \times 10^4 \left(\frac{M}{M_\odot}\right)L_\odot\)}
}

\newglossaryentry{RSG}{
name={RSG},
description={Red-supergiant \\ Supergaint stars with a spectral class of M or K  }
}

\newglossaryentry{ONeMg}{
name={ONeMg},
description={Fill this in later hopefully. Pretty sure it just means Oxygen, Neon and Magnesium?}
}

\newglossaryentry{chandrasekhar-limit}{
name={Chandrasekhar Limit},
description={Maxiumum mass of a white dwarf, approx \(M \approx 1.4M_\odot\)\\ This is support be electron degeneracy pressure}
}

\newglossaryentry{magnetic-breaking}{
name={magnetic breaking},
description={the process of a star magnetic field binding with the winds (or gas, etc), causing the stars spin to slow due to consv of ang momentum}
}

\newglossaryentry{LVB}{
name = {LVB},
description = {luminos blue variables.\\ These are which \(T_{eff} > 20000\text{K}\) }
}

\newglossaryentry{AGB}{
name = {AGB},
description = {Asymptontic Giant Branch. \\Part of the HR diagram where stars evolve off MS and into gaints}
}

\newglossaryentry{runaway-RLOF}{
name={runaway RLOF},
description={Unstable RLOF mass transfer which leads to eventual mergers}
}

\newglossaryentry{stellar-streams}{
name = {Stellar streams},
description = {Stars follwing a path away from a galaxy}
}

\newglossaryentry{surf-of-last-scatterting}{
name={Surface of last scattering},
description={Point looking deep into the universe where the universe becomes opaque. Also where the CMB is}
}

\newglossaryentry{ac:AIC}{
name={AIC},
description={Accretion Induced Collaspe \\ When a WD reaches the \gls{chandrasekhar-limit} in a close binary}
}

\newglossaryentry{natal-kick}{
name={Natal Kick},
description={A `kick' imparted on the object born from a SN from the SN itself. This is due to asymetric massloss, causing the star to be `pushed' in a direction}
}