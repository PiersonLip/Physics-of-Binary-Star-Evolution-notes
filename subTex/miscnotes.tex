\chapter{Misc}
\hrule
\vspace{.5cm}
\section{WD SNe}\label{WD-SNe}
\hrule
\vspace{.5cm}
\begin{list}{-}{}
\item This, simply-ish put, is caused by because a WD cannot reach hydrostatic equilibrium. A pressure increase does \textbf{not} lead to increase and radius, and thus cooling, instead, it just increases heat, and thus fusion, quickly getting out of hand. Can b trigged if the \gls{chandrasekhar-limit} is crossed, as well as if a layer of H/He on top entities, called a \gls{edge-limit-detonation} or ``sub-Chandrasekhar'' explosions.
\item In \gls{DD} systems the WDs must have a small separation, causing \gls{GWR} to shrink the separation until merger \(\rightarrow\) SNe
\end{list}

\section{Pauli Exclusion Principle}\label{pauli-exclusion-principle}
\hrule
\vspace{.5cm}
The property of matter that two identical particles with half integer spins (fermions) cannot occupy the same quantum state. \\
\begin{quoteBox}
\url{https://en.wikipedia.org/wiki/Pauli_exclusion_principle} \\ 
``In a poly-electron atom it is impossible for any two electrons to have the same two values of all four of their quantum numbers, which are: n, the principal quantum number; \(\ell\), the azimuthal quantum number; \(m\ell\), the magnetic quantum number; and ms, the spin quantum number. For example, if two electrons reside in the same orbital, then their values of n, \(\ell\), and \(m\ell\) are equal. In that case, the two values of \(ms\)(spin) pair must be different. Since the only two possible values for the spin projection \(ms\)are +1/2 and -1/2, it follows that one electron must have \(ms\)= +1/2 and one \(ms\)= -1/2.''
\end{quoteBox}
This means that in systems of very dense matter, the system \textit{\textbf{{cannot reach homogeneity}}}, and thus will actually have an inherent pressure. This pressure can prevent collapse due to gravity in various high dense objects (compact objects in astronomy).

\pagebreak
\section{Relativistic Beaming}\label{sec:RelBeaming}
\hrule
\vspace{.5cm}

The process of an object becoming brighter when it is moving \textit{towards} the Earth. This is \textit{in conjunction} with Doppler shift. This brightness increase can be attributed to the fact that, for a star moving \textit{towards} the earth, from the perspective of the earth, a `clock' placed near the star would run \textit{faster} than one on earth. If photons are emitted at a set rate on this `clock' it would be faster, when scaling the star `clock' to the earth one, it would be faster, meaning that the interval between received photons would be faster, and thus the object would appear brighter.  

\section{Ellipsoidal Modulation}
\label{sec:ellipsoidalModulation}
\hrule
\vspace{.5cm}
See \url{https://agn.caltech.edu/~srk/Ay215/Presentations/ellipsoidal_mod.pdf}!!!


