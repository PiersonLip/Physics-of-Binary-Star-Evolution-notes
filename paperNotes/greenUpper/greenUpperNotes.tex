\pagebreak
\chapter{An upper limit on the frequency of short-period black hole companions to Sun-like stars \cite{green_upper_2025}}
\hrule
\vspace{.5cm}
\section{Abstract}
\hrule
\vspace{.5cm}
\begin{list}{-}{}
\item Seeking to narrow the uncertainty regarding the survival rate in the formation of stellar-mass BHs. 
\item This uncertainty is most extreme in LMXB systems
\item Defines `close' systems as  \(P_{orb} \lesssim 3\text{d}\) 
\item Searching through for AFGK-type stars in \gls{TESS}\footnote{I wonder why TESS was picked}. These are the \textit{\textbf{non-accreting}} companion and progenitors of LMXBs
\item Detecting the BHs via the deformation of the main star via the light curves 
\item Found a selection of 457 candidates from a sample of \(4.7 \times 10^6\)
\item However, after spectra follow-up on 200 of the sample, \textit{none showed spectra consistent with a BH companion}\footnote{Very curious what ``spectra consistent with a BH companion'' means, especially as these are non accreting. Is this using a radial velocity method?\dots}
\item On the basis on non-detection they conclude that fewer than one in \(10^5\) \textit{sun-type} stars host a \textit{close} BH companion
\item This upper limit is in tension with that some models predict (\(\sim10^{4-5}\)), but congruent with other models that predict on in \(10^{7-8}\)
\end{list}

\pagebreak
\section{Introduction}
\hrule
\vspace{.5cm}
\begin{list}{-}{}
\item Isolated stars with mass greater than \(\gtrsim 15-25 M_\odot\) are believed to end their lives as BHs, suggesting that there are \(\sim 10^8\) stellar mass BHs in our galaxy. The majority of which are believed to evolve with a stellar companion, half of which will interact with said companion at some point during their lifespan.
\item A very small number (7) \gls{BH_LS} systems have been detected
\item Because of the scarcity of these systems, it's useful to try and set an upper limit on their distribution
\end{list}